\subsection{Intégrales dépendant d'un paramètre}

\begin{theorem}[Continuité sous le signe intégrale]
	Soit $f : U \times E \to \R$ où $(U,D)$ est un espace métrique.\\
	On suppose:
	\begin{itemize}
		\item $\forall u \in U,\ f (u, \cdot) : (E, \triA) \to (\R, \bor(\R))$ mesurable.
		\item $\mu-pp$ (en $x$) $f(\cdot, x) : U  \to \R$ est continue en $u_0 \in U$.
		\item Il existe $g$ intégrable telle que
		      $$\forall u \in U, \ \left| f(u,x) \right| \leq g(x) \mu-pp \text{ en } x$$
	\end{itemize}
	Alors la fonction
	$$F (u) = \int f(u,x) d\mu(x)$$
	est bien définie pour tout $u \in U$ et elle est continue en $u_0 \in U$.
\end{theorem}


\begin{proof}
	Il faut montrer que si $u_n$ et une suite avec $u_n \to u_0, \ n > 0$, alors $F(u_n) \to F(u_0)$.\\
	F est bien définie car $\forall u \in U, f(u,\cdot)$ mesurable et $|f(u,\cdot)| \leq g(\cdot)$ qui est intégrable.\\

	Posons $f_n(x) = f(u_n,x)$ et $f_n(x) \to f(u_0,x)$.\\
	$$|f_n(x)|= |f(u_n, x)| \leq g(x) \mu-pp$$
	D'après le TCD , $\underbrace{\int f_n(x)d\mu(x)}_{=F(u_n)} \to  \underbrace{\int f(u_0, x)d\mu(x)}_{= F(u_0)}$
\end{proof}

\begin{example}
	Soit $\mu$ une mesure diffuse sur $(\R, \bor(\R))$ et $\phi \in \Li(\R, \bor(\R), \mu)$.
	On définit \begin{eqnarray*}
		F(u) &=&  \int_{]-\infty, u]} \phi(x)d\mu(x)\\
		& = & \int \phi \1_{]-\infty, u]}(x)d\mu(x)
	\end{eqnarray*}
	alors $F$ est continue.

	\begin{proof}
		On pose $f(u,x) = \phi(x)\1_{]-\infty, u]}(x)$
		Il suffit de vérifier que $f(\cdot, x)$ est continue et qu'elle est dominée. Le reste est trivial: \\
		\begin{itemize}
			\item $u \mapsto f(u,x) = \left\{
				      \begin{array}{l}
					      \phi(x) \text{ si } u \geq x \\
					      0 \text{ sinon}
				      \end{array}
				      \right.$ \\
			      est continue en $\R\setminus\{x\}$. Comme $\mu$ est diffuse, $\mu \{x\} = 0$ et donc $f(\cdot, x)$ est continue $\mu-pp$.
			\item $|f(u,x)| = |\phi(x)\1_{]-\infty, u]}(x)| \leq |\phi(x)|$ qui est intégrable.
		\end{itemize}
	\end{proof}
\end{example}


\begin{theorem}[Dérivation sous le signe intégrale]
	On suppose que $U = I$ est un intervalle ouvert de $\R$ et $u_0 \in I$.\\
	$$ f: I\times E \to \R$$
	\begin{itemize}
		\item $\forall u \in I, \ f(u, \cdot) \in \Li_R(E,\triA, \mu)$
		\item $\mu-pp,\  f(\cdot, x)$ est dérivable en $u_0 \in I$ de dérivée $\diffp{f}{u} (u_0, x)$.
		\item Il existe $g\in \Li$ telle que
		      $$ \forall u\in I |f(u,x) - f(u_0,x)| \leq g(x)|u-u_0| \mu-pp$$
	\end{itemize}

	Alors $F(u) = \int f(u,x)d\mu$ est dérivable a point $u_0$ et sa dérivée est $F'(u_0) = \int \diffp{f}{u}(u_0,x)d\mu(x)$
\end{theorem}

\begin{remarque}
	La fonction $\diffp{f}{u}(u_0,\cdot)$ n'est définie que $\mu-pp$. Il suffit de
	la prolonger n'importe comment et cela suffit a définir $\int \diffp{f}{u}(u_0,x)d\mu(x)$.
\end{remarque}

\begin{proof}
	Soit $u_n \to u_0, \ n > 0, \ u_n \neq u_0$. \\
	On regarde
	\begin{eqnarray*}
		\frac{F(u_n)-F(u_0)}{u_n-u_0} &=& \frac{1}{u_n-u_0} \int f(u_n,x) -(u_0,x)df\mu(x) \\
		&=& \int \frac{f(u_n,x) - f(u_0,x)}{u_n-u_0}d\mu(x)
	\end{eqnarray*}
	\begin{itemize}
		\item $\frac{f(u_n,x)- f(u_0,x)}{u_n-u_0} \to \diffp{f}{u}(u_0,x)$
		\item $\left|\frac{f(u_n,x)- f(u_0,x)}{u_n-u_0} \right| \leq g(x)$
	\end{itemize}
	Donc d'après le TCD on a %TDOD: Add ref
	$$ \frac{F(u_n)- F(u_0)}{u_n-u_0} \to \int \diffp{f}{u}(u_0,x) d\mu(x)$$

\end{proof}

\begin{remarque}
	On peu changer les hypothèses 2 et 3 pour avoir une forme plus pratique:
	\begin{itemize}
		\item $\mu-pp\  f(\cdot, x)$ est dérivable sur $I$.
		\item Il existe $g\in \Li$ telle que
		      $$ \forall u\in I \left|\diffp{f}{u}(u,x)\right| \leq g(x) \mu-pp$$
		      Ceci implique, par le théorème des accroissements finis, la majoration de l'hypothèse 3.
	\end{itemize}
	et dans e cas on a que $F$ est dérivable sur $I$ tout entier.
\end{remarque}

\begin{remarque}
	Si $f$ est à valeurs complexes cela marche aussi.
\end{remarque}


\begin{example}[Trasformée de Fourier]
	Si $\phi \in \Li$ on définie sa transformée de Fourier:
	$\hat{\phi} = \int e^{iux}\phi(x) d\lambda(x)$
	alors $\hat{\phi}$ est bien définie dans $\R$ et continue (par le théorème de continuité sous le signe intégrale).\\
	Si de plus on a $\int |x\phi(x)|d\lambda(x)<\infty$ alors $\hat{\phi}$ est dérivable sur $\R$ de dérivée:
	$$\hat{\phi}'(x)= \int ixe^{iux}\phi(x)d\lambda(x)$$
	%TODO add proof
\end{example}



\begin{theorem}[Régularité $C^k$ sous le signe intégrale]
	Soit $f: I \times E \to \R$ où $I$ est un intervalle ouvert de $\R$\\

	\begin{itemize}
		\item $\forall u \in I, \ f(u,\cdot) \in \Li(E,\triA,\mu)$
		\item $\mu-pp u \ \mapsto f(u,x)$ est $C^k$ sur $I$.
		\item Il existe $g_k\in \Li$ telle que
		      $$ \forall u\in I \left|\diffp[i]{f}{u}(u,x)\right| \leq g_k(x) \mu-pp$$
	\end{itemize}

	Alors $F(u) = \int f(u,x)d\mu$ est $C^k$ sur $I$, avec $F^{(i)}(u) = \int \diffp[i]{f}{u}(u,x)d\mu(x)$

\end{theorem}
