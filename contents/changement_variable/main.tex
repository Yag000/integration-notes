
\section{La formule de changement de variables}


\begin{prop}[Formule de changement de variable linéaire]
	Soit $b \in \R^d, \ M \in \mathcal{M}_d(\R)$ inversible et
	$$ \begin{array}{rcl}
			f : \R^d & \longrightarrow & \R^d   \\
			x        & \longmapsto     & Mx + b
		\end{array} $$
	Alors, pour tout borélien $A \subset \R^d$,
	$$ \lambda_d(f(A)) = |\det(M)| \lambda_d(A) $$
\end{prop}

\begin{remarque}
	Si $M$ n'est pas inversible, $\lambda_d(f(A)) = 0$, car l'image est incluse dans un hyperplan $H$
	de $\R^d$, qui est donc de mesure nulle.
\end{remarque}

\begin{proof}
	On admet que la mesure de Lebesgue est invariante par translation et c'est la seule a multiplication près (exo). On peut donc supposer
	$b = 0$. Soit $A \subset \R^d$ borélien.

	$B \mapsto \lambda_d(f(B))$ est une mesure qui est invariante par translation.

	$f (\emptyset) = \emptyset$ et $f$ est bijective, donc l'image d'un d'une famille deux à deux disjoints est aussi deux à deux disjoints, et donc elle est
	$\sigma$-additive.
	\begin{eqnarray*}
		\lambda_d(f([B+a])) &=& \lambda_d(f(B) + f(a)) \\
		&=& \lambda_d(f(B))
	\end{eqnarray*}
	alors elle est un multiple de $\lambda_d$.

	Il suffit de vérifier que $\lambda_d(f([0,1]^d)) = |\det(M)|$.

	\begin{itemize}
		\item Si $M$ est diagonale $M = \begin{pmatrix} a_1 & & \\ & \ddots & \\ & & a_d \end{pmatrix}$, alors
		      $$ \lambda_d(f([0,1]^d)) = \lambda_d([0, a_1] \times \cdots \times [0, a_d]) =\left| \prod_{i=1}^d a_i \right|= |\det(M)| $$
		\item Si $M$ est orthogonale, le coefficient vaut 1 car $M$ conserve la boule unité.
		\item Si M est symétrique définie positive, le coefficient vaut $\det(M)$ car $M$ est diagonale après un changement de base.
		\item Dans le cas, $M = PS$ avec $P$ orthogonale et $S$ symétrique définie positive, avec $S =\sqrt{M^tM}$ et $P = M S^{-1}$.
		      Donc le coefficient vaut $|\det(P)| |\det(S)| = |\det(M)|$.
	\end{itemize}
\end{proof}


\begin{rappel}
	$U, D$ ouverts de $\R^d$, $\phi : U \to D$ est un $C^1$-difféomorphisme si $\phi$ est bijective et $C^1$ sur $U$ et $\phi^{-1}$ est $C^1$ sur $D$.
	Dans ce cas $\forall u \in U, \ \phi'(u)$ est inversible.
\end{rappel}


\begin{theorem}[Changement de variables]
	Soit $\phi: U \to D$ un $C^1$-difféomorphisme alors pour toute fonction $f : D \to \R^+$mesurable positive:
	$$\int_D f(x) d\lambda(x) = \int_U f(\phi(u)) |J_\phi (u)| d \lambda (u) $$
	où $J_\phi =\det(\phi(u))$ est le jacobien de $\phi$ en $u$.
\end{theorem}


