\subsection{Théorème de représentation de Riesz}

\begin{theorem}[Théorème de représentation de Riesz]
	Soit $H$ un espace de Hilbert, $\phi : H \to \C$ une forme \textbf{linéaire continue}.\\
	Alors il existe un unique $y \in H$ tel que $\forall x \in H, \phi(x) = \sprod{x}{y}$.
	De plus $\left|\norm{\phi}\right| = \norm{y}$.
\end{theorem}


\begin{proof}
	Si $\phi = 0$ alors $y = 0$ convient.\\
	Si ce n'est pas le cas on regarde $F = \phi^{-1}(\set 0)$, $\phi$ est continue donc $F$ est fermé.\\
	$F \subsetneq H$ car $\phi \neq 0$.\\
	En fait $\dim F^\perp = 1$\\
	Si $x,y \in F^\perp,\ x \neq 0, y \neq 0$ alors $\sprod{x}{y} \neq 0$ car $x \neq 0$ et $y \neq 0$.\\
	%TODO
\end{proof}


